\chapter{Introdução}\label{Introducao}

É notório o aumento da preocupação com a segurança nos dias atuais, com as cidades ficando cada vez mais violentas,  buscas por soluções são eminentes. A tecnologia, por sua vez, vem com o intuito de proporcionar a comodidade, segurança e facilidade, com propostas de acesso remoto a dispositivos de automação residencial, que proporcionam ao usuário o controle da sua casa, assim como seus bem, dentro e fora de casa \citep{Wortmeyer2005}.

\citep{Silva2016}, em seu trabalho, expõe a importância de um sistema de automação comercial com o exemplo de controle para retirada de carrinhos de bebês em um shopping de Salvador, onde mostra a ineficiência de um controle manual. Com o sistema desenvolvido, com uso de tecnologias como \acf{RFID}, foi possível ter o controle dos carrinhos retirados, tal como sua localização, caso o mesmo não for devolvido. Podemos verificar que a eficácia da junção do software com hardware para trazer facilidade e praticidade para o meio comercial.

A Embrapa Informática Agropecuária, com intuito de ter o controle dos bens patrimoniais, investiu na tecnologia RFID, uma vez que verificações manuais não são totalmente confiáveis. Juntamente com uma aplicação Web, tornou de mais fácil acesso e controle os bens tal como sua localização. Por meio das tags e/ou etiquetas RFID, é possível identificar unicamente cada bem, assim como sua localização, também tornando possível emissões de relatórios sempre que necessário \citep{Narciso2008}.


\section{Justificativa}

Atualmente o controle de acesso aos laboratórios de informática do IFG é realizado de forma manual, para ter acesso aos laboratórios basta preencher manualmente a folha de controle de chaves no \acf{DAA} e retirar a chave. Essa prática permite que pessoas não autorizadas acessem livremente o laboratório, comprometendo a segurança física do ambiente. Levando em consideração o alto valor agregado (computadores, roteadores, switches e outros equipamentos), considera-se que seja importante o desenvolvimento de um sistema que garanta maior segurança no acesso físico desses ambientes.

Sabendo que o RFID é uma tecnologia que captura dados por meio de frequência de rádio, tornando possível a identificação de objetos com dispositivos eletrônicos (tags) e que o Arduíno, juntamente com o ESP-01, podem fazer a leitura destes sinais e envia-los via \acf{WiFi} para uma página Web, tornando possível construir um sistema computacional baseado na Web para efetivo controle de acesso aos laboratórios do IFG utilizando o Arduíno, o ESP-01 e a tecnologia de comunicação RFID.

\section*{Objetivo}

\subsection{Objetivo Geral}

Sistematizar e projetar sistema de controle de acesso Web utilizando RFID e microcontrolador ESP8266 através de estudo bibliográfico.

\subsection{Objetivos Específicos}

\begin{enumerate}
    \item Estudar conceitos relacionados aos dispositivos embarcados presentes no contexto de IoT, tais como a plataforma de desenvolvimento livre e aberta Arduíno e placas baseadas no microcontrolador ESP8266 (ESP-01);
	
	\item Pesquisar sobre eletrônica básica relacionada à montagem de circuitos digitais através de uma \textit{protoboard};

	\item Prototipar o sistema a nível de software através de simulação;

	\item Projetar e construir bancos de dados relacional para cadastro de usuários do futuro sistema;  
	
	\item Realizar testagem física dos dispositivos a fim de testar os códigos criados para o Arduíno e o ESP-01.

\end{enumerate}



\section{Descrição dos Capítulos}

A última parte da introdução é uma descrição do que o leitor encontrará em cada capítulo.
Por exemplo, no \nameref{Referencial_Teorico} são apresentados comandos em \LaTeX e dicas para formatar seu TCC.
No \nameref{Metodo} é descrito o método utilizado para alcançar os objetivos listados aqui.
O \nameref{Resultados} apresenta os resultados obtidos.
No \nameref{Conclusao}, são apresentadas as conclusões, contribuições científicas e trabalhos futuros.




