\chapter{Introdução}\label{Introducao}

Este documento serve de exemplo da utilização da classe IFG para escrever um texto cujo objetivo é apresentar os resultados de um trabalho científico. 
Este modelo foi produzido a partir do modelo de monografia do \href{https://github.com/UnB-CIC/Monografia}{Departamento de Ciência da Computação da UnB}.

Na introdução, o autor deve apresentar os fatos e fatores que influenciam o contexto do trabalho.
Em seguida, o autor deve mostrar seu problema nesse contexto em um último parágrafo.
Os objetivos devem ser destacados em uma subseção.
Citações são mais do que uma obrigação, são o reconhecimento e respeito pelo trabalho dos nossos colegas de ciência.

Para citar assim:~\citep{Silva2018graph}, Escreva assim:

\begin{verbatim}
~\citep{Silva2018graph}
\end{verbatim}

Para citar assim:~\cite{Silva2018graph}, Escreva assim:

\begin{verbatim}
~\cite{Silva2018graph}
\end{verbatim}


\section*{Objetivo}

Objetivos normalmente iniciam-se com verbos no infinitivo, por exemplo: 

Escrever um bom TCC.

\subsection*{Objetivos Específicos}

\begin{enumerate}
	\item Fazer AAAA
	\item Implementar BBBB
	\item Produzir CCCC
\end{enumerate}

Alguns cuidados devem ser tomados para melhorar a experiência do leitor.
O primeiro deles é sempre usar referências cruzadas de siglas, capítulos e seções, figuras e tabelas.
O nome a seguir: \acf{IFG}, é um exemplo de uso de siglas para ser usado na primeira citação.
\ac{IFG} é apenas a sigla, após seu primeiro uso.
Este é um exeplo de nota de rodapé\footnote{\url{https://www.capes.gov.br/images/stories/download/diversos/OrientacoesCapes_CombateAoPlagio.pdf}}.

\section{Descrição Dos Capítulos}

A última parte da introdução é uma descrição do que o leitor encontrará em cada capítulo.
Por exemplo, no \nameref{Referencial_Teorico} são apresentados comandos em \LaTeX e dicas para formatar seu TCC.
No \nameref{Metodo} é descrito o método utilizado para alcançar os objetivos listados aqui.
O \nameref{Resultados} apresenta os resultados obtidos.
No \nameref{Conclusao}, são apresentadas as conclusões, contribuições científicas e trabalhos futuros.


